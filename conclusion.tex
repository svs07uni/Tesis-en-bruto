\label{conclusiones}
\chapter{Conclusiones}
Gukena logró un cambio notable en las elecciones de la Universidad Nacional del Comahue, tanto a nivel de velocidad del escrutinio como de claridad y transparencia en los resultados finales. Además distribuyó la carga de trabajo a un grupo de personas pero manteniendo la responsabilidad de verificación a una persona encargada, quien antes de Gukena debía encargarse tanto de la carga como el proceso y generar los resultados finales. Como se observó en las experiencias del sistema en cada año, con una mínima incorporación de tecnología en el proceso electoral se logra cubrir las propiedades analizadas en el capítulo 2. Además gracias a estas experiencias exitosas se planteó a nivel provincial un cambio de protocolo en las elecciones electrónicas (BUE) en la Junta Electoral Provincial, con el objetivo de validar de manera aleatoria las máquinas utilizadas, agregando una verificación más dentro del proceso electoral.


Trabajos Futuros - Boleta única en la Unco


