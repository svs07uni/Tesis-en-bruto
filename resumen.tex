\ \\
\ \\
\label{pagresum}
\noindent{\LARGE \sc Resumen}\\
\ \\
\ \\

\ \\

\ \\
\ \\

 En estos tiempos, donde la tecnología forma parte de nuestra rutina y en muchos casos se confía plenamente en ella, podemos decir que nos ayuda a automatizar múltiples procesos y reduce los tiempos de nuestras tareas. En los últimos años se ha introducido tecnología en procesos electorales del ámbito universitario, municipal, provincial, nacional e internacional. Los sistemas electorales se pueden definir como sistemas críticos donde una falla puede ser irreversible y con un gran costo. Las leyes Argentinas imponen que el voto sea universal, igualitario, secreto y obligatorio. Un sistema de votación exitoso debe lograr la confianza de toda persona involucrada en el proceso: ciudadanos, partidos políticos y gobierno. Por este motivo, la voluntad del elector es el dato más importante en este sistema, una falla en su conteo o no procesar correctamente el voto emitido puede generar errores que se propagan en el escrutinio de todo el comicio y puede darnos un resultado equivocado.\newline
 El proceso electoral comienza cuando se definen las fechas de oficialización de padrones y listas, pasando luego por el acto electoral, escrutinio y proclamación de autoridades elegidas. Por lo tanto, una falla en cualquiera de estas etapas provocaría que se pierda la confianza en todo el proceso.\newline
 En esta tesis se evalúan las propiedades que deben protegerse dentro de un sistema electoral y cómo impacta la tecnología a cada una de ellas, analizando experiencias de elecciones a nivel municipal e internacional. A partir de esta evaluación se presenta un sistema que mantiene la emisión del voto en papel de cada elector, y el escrutinio manual por parte de las autoridades de mesa. Con el objetivo de mejorar los tiempos del proceso de cómputo, sin poner en riesgo la confianza del sistema y garantizar el secreto del voto, se desarrolla la propuesta de ``acta electrónica'' denominada Gukena, donde cada autoridad de mesa luego de su escrutinio manual realiza la carga de esta información.\newline
 Gukena ha sido utilizado en el ámbito de la Universidad Nacional del Comahue a partir del 2015. Este sistema resolvió el cómputo de los votos que, previo a esta fecha, se realizaba manualmente por una única persona encargada de ingresar los datos, gestionar correctamente los resultados y distribuir los cargos. La forma de trabajo generaba un cuello de botella en la carga de los datos y una demora de varias horas y hasta días en obtener los resultados finales.\newline
 Esta herramienta permitió agilizar el escrutinio final debido a la carga descentralizada y sin afectar la claridad de la información enviada. Se logró reflejar los resultados provisorios de todas las mesas en el instante en que cada autoridad de mesa enviaba los datos. \newline
 %\cambiar{Por último en este trabajo se realiza una comparación de velocidad de carga de las elecciones provinciales de Río Negro, Neuquén y Córdoba en relación con el sistema Gukena.}
 Por último en este trabajo se realiza una comparación de velocidad de carga entre las elecciones provinciales de Río Negro, Neuquén y Córdoba en relación con el sistema Gukena. 

\vfill
\pagebreak
