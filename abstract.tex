\ \\
\ \\
\label{pagsumm}
\noindent{\LARGE \sc Abstract}\\
\ \\
\ \\

\ \\

\ \\
\ \\

Currently, the technology is part of our routine and in many cases it's fully trusted, we can say that it helps us to automate multiple processes and reduce the time to do our tasks. In recent years, technology has been introduced into university, municipal, provincial, national and international elections. Electoral systems are crytical systems where a failure can be irreversible and costly. Argentine law requires that the vote be universal, equal, secret and compulsory. A successful voting system must have the trust of everyone involved in the process: citizens, political parties and government. For this reason, the voter's will is the most important element in this system, a mistake in counting it or a wrong processing of it could provoke errors that can propagate in the scrutiny of the whole election and give us a wrong result.\newline
The electoral process initiates when the officialization of the lists of voters and the electoral rolls are defined, throughing the electoral act, scrutiny and the proclamation of the elected authorities. Therefore, a failure in any of these steps could make the entire process become trustless.\newline
This thesis evaluates the properties that must be protected inside an electoral system and how technology impacts each of these properties, analyzing experiences of elections at a municipal and international level. After this evaluation, a system is presented that maintains the paper ballot emission of each voter, and a manual scrutiny by the authorities of the electoral table. In order to improve the time of the counting process, without the risk of loose a system confidence and to guarantee the secret of the vote, the proposal of ``electronic certificate'' called Gukena has been developed, where each table authority, after the manual counting of votes, submitted this information.\newline
%\cambiar{distributing the roles por distibuting the positions  |  distributing the appointments }
Gukena has been used at the National University of Comahue since 2015. This system resolve the counting of votes, which before this year, it was doing manually by a single person who was responsible for registering the information, managing the results correctly and distributing the positions. This way of working generated a bottleneck in filling the data and it took several hours or even days to get the final results.\newline
%\cambiar{Provisional results por Interim results }
This system helped to reduce the time required for the final scrutiny due to the decentralized uploading and without affecting the transparency of the information submitted. Interim results were reflected as soon as each voting table authority submitted the information.\newline
Finally, this document compares the scrutiny velocity between the provincial elections of Río Negro, Neuquén and Córdoba in relation to the Gukena system.

\vfill
\pagebreak
