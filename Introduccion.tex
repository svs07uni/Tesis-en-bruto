\label{Introduccion}
\chapter{Introducción}
El sistema electoral argentino debe asegurar que la votación sea universal, igual, secreto y obligatorio \citep{arlettaz2012libertad}. El proceso electoral comienza cuando se definen las fechas de oficialización de padrones y listas, acto electoral, escrutinio provisorio, escrutinio definitivo y proclamación de autoridades electas. Se considera que un sistema de votación exitoso debe cumplir el principal objetivo que es construir la confianza de toda persona involucrada en el proceso: ciudadanos, partidos políticos, gobierno, entre otros. Como posibles alternativas en este proceso, planteadas a nivel mundial, ha sido aplicar distintas técnicas para computarizar parte del proceso de votación. Existen diferentes dominios con antecedentes a estas técnicas, tanto en Argentina, como en  otros  países,  con  distintos  sistemas  de  voto  electrónico. Tomando estas experiencias y el objetivo principal que debe cumplir un sistema electoral, se parte de la idea de generar un sistema que mantenga la separación física del votante con la tecnología, y forzar un escrutinio manual por parte de las autoridades de mesa en el momento del conteo. \newline
En este capítulo se presenta la motivación, objetivos y organización de la tesis.

\section{Motivación}
Antes del 2015, el cómputo provisorio de los votos en la Universidad Nacional del Comahue se realizaba en forma manual por una persona, encargada de ingresar los datos en una planilla electrónica. Esta persona era responsable de gestionar correctamente los resultados y distribuir los cargos. La forma de trabajo generaba un cuello de botella en la carga de los datos produciendo una demora de varias horas y hasta días en obtener los resultados finales. \newline
Con esto, se desarrolla la propuesta de ``acta electrónica'' basada en el proceso de carga electrónica del acta creada en papel, mediante un método descentralizado  por parte de cada autoridad de mesa. Dicha herramienta evitaría el contacto del votante con un sistema electrónico, resguardando el anonimato del elector, y ayudaría a cada autoridad de mesa el envío de datos de manera rápida y transparente. De todos modos, el sistema no quita el envío de telegramas para validar la consistencia de los datos con su respectiva acta, por lo tanto existen validaciones que permite corregir cualquier desvío en los datos cargados en el sistema. Este sistema permitiría disminuir la demora del escrutinio final, sin afectar la claridad de la información enviada. Se reflejan los resultados provisorios de todas las mesas al instante que cada autoridad de mesa envía los datos.

\section{Objetivos de la Tesis}
El objetivo de la tesis es el diseño e implementación de un sistema de acta electrónica para el escrutinio descentralizado de elecciones. Con esto se pretende agilizar el recuento de votos y lograr un resultado provisorio disponible en tiempo real. Se analizará el impacto del uso de la tecnología en relación con el sistema electoral tradicional, evaluando experiencias en distintos ámbitos tanto con éxito como fracaso, permitiendo analizar y evaluar características que ayudaron o fracasaron dentro de un sistema real.\newline
Dicha herramienta permitirá cargar los datos obtenidos en cada punto de votación y luego validar su consistencia mediante la interacción con autoridades electorales. 

\section{Organización de la Tesis}
A continuación se describe la estructura de esta tesis:
\begin{enumerate}
    \item Capítulo \ref{SistemaElectoral}: revisión conceptual sobre las características y modelo sobre un sistema electoral tradicional.
    \item Capítulo \ref{Elecciones}: revisión sistemática de la utilización de tecnología en  Elecciones en diferentes dominios.
    \item Capítulo \ref{Gukena}: se describe el sistema desarrollado Gukena como opción en la utilización de las elecciones de la Universidad Nacional del Comahue.
    \item Capítulo \ref{Gukena2}: se exponen los resultados obtenidos luego de utilizarse Gukena en las diferentes elecciones realizadas en la Universidad Nacional del Comahue entre el 2015 y 2019.
\end{enumerate}


\agregar{Metodologia}