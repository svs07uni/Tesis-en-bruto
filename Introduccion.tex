\label{Introduccion}
\chapter{Introducción}
Como posibles soluciones, planteadas a nivel mundial, ha sido aplicar distintas técnicas para computarizar parte del proceso de votación. En estos casos, se considera que un sistema de votación exitoso debe cumplir el principal objetivo que es construir la confianza de toda persona involucrada en el proceso: ciudadanos, partidos políticos, gobierno, entre otros. El sistema electoral argentino debe asegurar que la votación sea universal, igual, secreto y obligatorio \citep{votoprimeravez}. Cabe mencionar que existen antecedentes, tanto en Argentina, como en  otros  países,  de  distintos  sistemas  de  voto  electrónico. Tomando estas experiencias, se parte de la idea de generar un sistema que mantenga la separación física del votante con la tecnología, y forzar un escrutinio manual por parte de las autoridades de mesa en el momento del conteo. \newline
En este capítulo se presenta la motivación, objetivos y organización de la tesis.
\agregar{En función de cualquiera de los articulos gukena \url{http://47jaiio.sadio.org.ar/sites/default/files/SIE-05.PDF}}

\section{Motivación}
El cómputo provisorio de los votos en la Universidad Nacional del Comahue se realizaba en forma manual por una persona, encargada de ingresar los datos en una planilla electrónica. Esta persona era responsable de gestionar correctamente los resultados y distribuir los cargos. La forma de trabajo generaba un cuello de botella en la carga de los datos produciendo una demora de varias horas y hasta días en obtener los resultados finales. \newline
Con esto, se desarrolla la propuesta de ''acta electrónica'' basada en el proceso de carga electrónica del acta creada en papel, mediante un método descentralizado  por parte de cada autoridad de mesa. Dicha herramienta evitaría el contacto del votante con un sistema electrónico, resguardando el anonimato del elector, y ayudaría a cada autoridad de mesa el envío de datos de manera rápida y transparente. De todos modos, el sistema no quita el envío de telegramas para validar la consistencia de los datos con su respectiva acta. Esta técnica permitiría disminuir la demora del escrutinio final, sin afectar la claridad de la información enviada. Este sistema refleja los resultados provisorios de todas las mesas al instante que cada autoridad de mesa envía los datos.

\section{Objetivos de la Tesis}
El objetivo de la tesis es el diseño e implementación de un sistema de acta electrónica para el escrutinio descentralizado de elecciones. Con esto se pretende agilizar el recuento de votos y lograr un resultado provisorio disponible en tiempo real. De igual modo, se analizará el impacto del uso de la tecnología en relación con el sistema electoral tradicional, evaluando casos de estudios desarrollados tanto con éxito como fracaso.\newline
Dicha herramienta permitirá cargar los datos obtenidos en cada punto de votación y luego validar su consistencia mediante la interacción con autoridades electorales. 

\section{Organización de la Tesis}
El siguiente capítulo de las características que tienen los sistemas electorales.....,

El capítulo \ref{Elecciones} se realiza una revisión sistemática de la utilización de tecnología en  Elecciones en diferentes Dominios.\\

En el capítulo \ref{Gukena} se describe el desarrollo del sistema Gukena .....\\

En el capítulo \ref{Gukena2} se anlizan las experiencias de uso del sistema en las elecciónes entre el 2015 y 2019.....

\agregar{Metodologia}