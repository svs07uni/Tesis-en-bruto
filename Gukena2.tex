\label{Gukena2}
\chapter{Elecciones Uncoma con Gukena}

\section{Experiencia 16 de Junio de 2015}
Para las elecciones celebradas el 16 de Junio de 2015, Gukena se validó con el archivo xls usado en el escrutinio. Este año tuvo información de 18 Unidades Electorales distribuidas en 16 localidades con un total de 34 mesas. 
Ese año, la Universidad renovó integrantes del Consejo Directivo de cada unidad académica y del Consejo Superior de la Universidad, para el claustro de Estudiantes.

De este modo, el sistema se puso a prueba frente a la siguiente información:
\begin{itemize}
    \item La Elección se desarrolló en 18 Unidades Electorales repartidas en 34 Sedes.
    \item 34 mesas distribuidas en las distintas sedes
    \item 9.560 empadronados del claustro Estudiantes
    \item 9 listas para el Consejo Superior
    \item 45 listas para el Consejo Directivo distribuidas entre las diferentes sedes y asentamientos
\end{itemize}
A pesar de que este año no se utilizó el sistema durante el escrutinio, los datos se encuentran disponible en las elecciones históricas presente en la página de Resultados de Gukena.

\section{Experiencia 14 al 17 de Mayo de 2016}
Para las elecciones celebradas desde el 14 al 17 de Mayo de 2016, Gukena contó con información de 18 Unidades Electorales distribuidas en 16 localidades con un total de 77 mesas. 
Ese año, la Universidad renovó integrantes del Consejo Directivo de cada unidad académica y del Consejo Superior de la Universidad, para los claustros de Estudiantes, No Docentes y Graduados. 

De este modo, el sistema operó frente a la siguiente información:
\begin{itemize}
    \item La Elección se desarrolla en 18 Unidades Electorales repartidas en 34 Sedes.
    \item De las 77 mesas distribuidas en las distintas sedes, 18 son para el claustro no docente, 33 son para el claustro estudiantes y 26 son para el claustro graduados.
    \item De 22.313 empadronados: 719 son No docentes, 8199 son Estudiantes y 13.395 son Graduados. 
    \item 15 listas para el Consejo Superior, donde 4 son de No docentes, 8 son de estudiantes y 3 de graduados.
    \item 121 listas para el Consejo Directivo distribuidas entre las diferentes sedes, asentamientos y distintos claustros.
\end{itemize}

Durante el período de escrutinio, la primer mesa cargada con éxito en el sistema fue la Facultad de Ciencias y Ambientes de la Salud, para el claustro de graduados, 5 minutos después del cierre.\\ 
Con la utilización del sistema, se obtuvo la siguiente información estadística:
66 mesas fueron cargadas directamente por autoridades de mesas siendo el 86\% del total
de estas 66 mesas, 39 mesas fueron cargadas entre las 20:00 hs y 21:06 hs, siendo el 51\% del total.\\
11 mesas fueron cargados por la junta electoral, de las cuales una fue corregida en una segunda etapa de verificación.
Sólo 3 valores de actas fueron corregidas por la junta electoral, lo que implica un 0,3\% de error en lo cargado por las autoridades de mesa, o lo que es lo mismo un 99,7\% de datos correctos cargados por las autoridades de mesas.

Las razones por las cuales 11 actas no pudieron ser cargadas por las Autoridades de Mesa son las siguientes:
\begin{itemize}
\item La persona destinada a cargar los datos no contaba con experiencia en el uso de la computadora.
\item El extravío del sobre que contenía datos de usuario y contraseña enviado junto a la urna.
\item Problemas con la conexión a Internet.
\end{itemize}

\section{Experiencia 13 al 16 de Mayo de 2017}
Gukena procesó información de 17 Unidades Electorales distribuidas en 14 localidades con un total de 32 mesas. Este año, la Universidad renovó integrantes del Consejo Directivo de cada Unidad Académica y el Consejo Superior en el claustro Estudiantes.

De este modo, el sistema operó frente a la siguiente información:
\begin{itemize}
    \item 32 mesas distribuidas en las distintas sedes
    \item El padrón fue de 9504 estudiantes.
    \item 8 listas para Consejo Superior.
    \item 38 listas para el Consejo Directivo entre las diferentes sedes y asentamientos.
\end{itemize}
  
Del total de mesas se tuvo que 31 mesas fueron cargadas por las autoridades de mesa desde la sede donde se realizó la elección y 1 fue cargada por la Junta Electoral. Todos los datos fueron correctamente cargados, esto quiere decir que ningún dato contenido en las actas debió ser corregido por la Junta Electoral, ni por la Secretaría de la misma.
Durante el período de escrutinio, la primer mesa cargada con éxito en el sistema fue la Facultad de Turismo sede San Martín de los Andes, 3 minutos después del cierre.

De la utilización del sistema, se obtuvo la siguiente información estadística:
31 mesas fueron cargadas directamente por autoridades de mesas, siendo el 97\% del total.

Del total de mesas:
15 mesas fueron cargadas entre las 20:00 hs y 21:20 hs, siendo el 47\%,
%20 mesas fueron cargadas entre las 20:00 hs y 21:40 hs, siendo el 63\%,
%22 mesas fueron cargadas entre las 20:00 hs y 21:50 hs, siendo el 69\% y 
%6 mesas fueron cargadas al día siguiente, siendo el 19\%.
1 mesa fue cargada 	por la Junta Electoral, siendo el 3\% del total.

Ningún acta fue corregida por la Junta Electoral, lo que implica un 0\% de error en la carga realizada por las autoridades de mesa, o lo que es lo mismo un 100\% de datos correctamente cargados por las autoridades de mesas.

\section{Experiencia 2018}
\subsection{Primer vuelta 21 al 22 de Mayo de 2018}
Gukena procesó información de 18 Unidades Electorales distribuidas en 14 localidades con un total de 95 mesas. Este año, la universidad renovó integrantes del Rectorado, Decanato, Consejo Directivo de cada unidad y el Consejo Superior de la UNCo en el claustro Estudiantes, Graduados, No Docentes y Docentes.

De este modo, el sistema operó frente a la siguiente información:
\begin{itemize}
    \item 95 mesas distribuidas en las distintas sedes 
     \item El padrón fue de  30192, distribuidos en 9486 estudiantes, 17481 graduados, 718 no docentes y 2507 docentes.
     \item 5 listas para Rectorado.
     \item 30 listas para Decano
    \item 23 listas para Consejo Superior, distribuidos en 7 del claustro Estudiantes, 5 de Graduado, 5 No Docente, 6 Docente.
    \item 133 listas para el Consejo Directivo: 35 del Claustro Estudiantes, 31 de Graduados, 23 No Docentes, Docentes 44.
\end{itemize}
Del total de mesas, 94 mesas fueron cargadas por las autoridades de mesa desde la sede donde se realizó la elección y 1 fue cargada por la Junta Electoral y, 90 mesas fueron correctamente cargados, esto quiere decir que 5 mesas debieron ser corregidos por la Junta Electoral o por la Secretaría de la misma.

Durante el período de escrutinio, la primer mesa cargada con éxito en el sistema fue la AUZA sede Zapala a las 19:38 hrs. debido a que al ser una mesa con muy poca concurrencia se habilitó la carga minutos antes de las 20:00 hrs.

De la utilización del sistema, se obtuvo la siguiente información estadística:
93 mesas fueron cargadas directamente por autoridades de mesas, siendo el 99\% del total.

Del total de mesas:
45 mesas fueron cargadas antes de las 21:30 hs, siendo el 47\%,
%20 mesas fueron cargadas entre las 20:00 hs y 21:40 hs, siendo el 63\%,
%22 mesas fueron cargadas entre las 20:00 hs y 21:50 hs, siendo el 69\% y 
%6 mesas fueron cargadas al día siguiente, siendo el 19\%.
1 mesa fue cargada 	por la Junta Electoral, siendo el 1\% del total.

5 actas fue corregida por la Junta Electoral, lo que implica un 5\% de error en la carga realizada por las autoridades de mesa, o lo que es lo mismo un 95\% de datos correctamente cargados por las autoridades de mesas.

\subsection{Segunda vuelta 4 al 5 de Junio de 2018}
Gukena procesó información de 18 Unidades Electorales distribuidas en 14 localidades con un total de 97 mesas. En esta segunda vuelta, la universidad disputó el Rectorado participando los claustros Estudiantes, Graduados, No Docentes y Docentes.

De este modo, el sistema operó frente a la siguiente información:
\begin{itemize}
    \item 97 mesas distribuidas en 26 de las 34 Sedes de la UnComa.
     \item El padrón fue de  30192, distribuidos en 9486 estudiantes, 17481 graduados, 718 no docentes y 2507 docentes.
     \item 5 listas para Rectorado.
\end{itemize}
Del total de mesas, 97 mesas fueron cargadas por las autoridades de mesa desde la sede donde se realizó la elección y 1 fue cargada por la Junta Electoral y, 97 mesas fueron correctamente cargados, esto quiere decir que ninguna mesa debió ser corregida por la Junta Electoral o por la Secretaría de la misma.

Durante el período de escrutinio, la primer mesa cargada con éxito en el sistema fue ASMA sede San Martin de los Andes a las 20:02 hrs.

De la utilización del sistema, se obtuvo la siguiente información estadística:
96 mesas fueron cargadas directamente por autoridades de mesas, siendo el 99\% del total.

Del total de mesas:
50 mesas fueron cargadas antes de las 20:40 hs, siendo el 52\%,
%20 mesas fueron cargadas entre las 20:00 hs y 21:40 hs, siendo el 63\%,
%22 mesas fueron cargadas entre las 20:00 hs y 21:50 hs, siendo el 69\% y 
%6 mesas fueron cargadas al día siguiente, siendo el 19\%.
1 mesa fue cargada 	por la Junta Electoral, siendo el 1\% del total.

Ningún acta fue corregida por la Junta Electoral, lo que implica un 0\% de error en la carga realizada por las autoridades de mesa, o lo que es lo mismo un 100\% de datos correctamente cargados por las autoridades de mesas.

\section{Experiencia 14 de Mayo de 2019}
Gukena procesó información de 18 Unidades Electorales distribuidas en 14 localidades con un total de 30 mesas. Este año, la Universidad renovó integrantes del Consejo Directivo de cada Unidad Académica y el Consejo Superior en el claustro Estudiantes.

De este modo, el sistema operó frente a la siguiente información:
\begin{itemize}
    \item 30 mesas distribuidas en las distintas sedes
     \item El padrón fue de  9283 estudiantes.
     \item 27 listas para el Consejo Directivo del Claustro Estudiantes.
     \item 7 listas para Consejo Superior
\end{itemize}
Del total de mesas, 30 mesas fueron cargadas por las autoridades de mesa desde la sede donde se realizó la elección y ninguna mesa fue cargada por la Junta Electoral y, ninguna mesa debió ser corregida por la Junta Electoral o por la Secretaría de la misma.

Durante el período de escrutinio, la primer mesa cargada con éxito en el sistema fue FAAS sede Puerto Madryn a las 20:01 hrs.

De la utilización del sistema, se obtuvo la siguiente información estadística:
30 mesas fueron cargadas directamente por autoridades de mesas, siendo el 100\% del total.

Del total de mesas:
15 mesas fueron cargadas antes de las 21:10 hs, siendo el 50\%,
%20 mesas fueron cargadas entre las 20:00 hs y 21:40 hs, siendo el 63\%,
%22 mesas fueron cargadas entre las 20:00 hs y 21:50 hs, siendo el 69\% y 
%6 mesas fueron cargadas al día siguiente, siendo el 19\%.

Ningún acta fue corregida por la Junta Electoral, lo que implica un 0\% de error en la carga realizada por las autoridades de mesa, o lo que es lo mismo un 100\% de datos correctamente cargados por las autoridades de mesas.

\section{Velocidad de carga}

