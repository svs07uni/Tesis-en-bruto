\label{Gukena2}
\chapter{Trabajo de Campo ... Experiencias Gukena}

\subsection{Experiencia 2015}
\subsection{Experiencia 14 al 17 de Mayo de 2016}
Para las elecciones celebradas desde el 14 al 17 de Mayo de 2016, Gukena contó con información de 18 Unidades Electorales distribuidas en 16 localidades con un total de 77 mesas. 
Ese año, la Universidad renovó integrantes del Consejo Directivo de cada unidad académica y del Consejo Superior de la Universidad, para los claustros de Estudiantes, No Docentes y Graduados. 

De este modo, el sistema operó frente a la siguiente información:
La Elección se desarrolla en 18 Unidades Electorales repartidas en 34 Sedes.
De las 77 mesas distribuidas en las distintas sedes, 18 son para el claustro no docente, 33 son para el claustro estudiantes y 26 son para el claustro graduados.
De 22.313 empadronados: 719 son No docentes, 8199 son Estudiantes y 13.395 son Graduados. %está bien este dato? más graduados que estudiantes?
15 listas para el Consejo Superior, donde 4 son de No docentes, 8 son de estudiantes y 3 de graduados.
121 listas para el Consejo Directivo distribuidas entre las diferentes sedes, asentamientos y distintos claustros.

Durante el período de escrutinio, la primer mesa cargada con éxito en el sistema fue la Facultad de Ciencias y Ambientes de la Salud, para el claustro de graduados, 5 minutos después del cierre.\\ 
Con la utilización del sistema, se obtuvo la siguiente información estadística:
66 mesas fueron cargadas directamente por autoridades de mesas siendo el 86\% del total
de estas 66 mesas, 39 mesas fueron cargadas entre las 20:00 hs y 21:06 hs, siendo el 51\% del total.\\
11 mesas fueron cargados por la junta electoral, de las cuales una fue corregida en una segunda etapa de verificación.
Sólo 3 valores de actas fueron corregidas por la junta electoral, lo que implica un 0,3\% de error en lo cargado por las autoridades de mesa, o lo que es lo mismo un 99,7\% de datos correctos cargados por las autoridades de mesas.
Las razones por las cuales 11 actas no pudieron ser cargadas por las Autoridades de Mesa son las siguientes:
\begin{itemize}
\item La persona destinada a cargar los datos no contaba con experiencia en el uso de la computadora.
\item El extravío del sobre que contenía datos de usuario y contraseña enviado junto a la urna.
\item Problemas con la conexión a Internet.
\end{itemize}

\subsection{Experiencia 13 al 16 de Mayo de 2017}
Gukena procesó información de 17 Unidades Electorales distribuidas en 14 localidades con un total de 32 mesas.
Este año, la Universidad renovó integrantes del Consejo Directivo de cada Unidad Académica y el Consejo Superior en el claustro Estudiantes.
%La UNCo se compone de 18 Unidades Electorales repartidas en 34 Sedes, donde las elecciones 2017 se llevaron a cabo en 26 de ellas.
El padrón fue de 9504 estudiantes.  Se presentaron 8 listas para Consejo Superior y 38  para Consejo Directivo.
La cantidad de mesas registradas en Gukena fue 32, de las cuales: 	
31 mesas fueron cargadas por las autoridades de mesa desde la sede donde se realizó la elección y 1 fue cargada por la Junta Electoral. Todos los datos fueron correctamente cargados, esto quiere decir que ningún dato contenido en las actas debió ser corregido por la Junta Electoral, ni por la Secretaría de la misma.
Durante el período de escrutinio, la primer mesa cargada con éxito en el sistema fue la Facultad de Turismo sede San Martín de los Andes, 3 minutos después del cierre.

De la utilización del sistema, se obtuvo la siguiente información estadística:
31 mesas fueron cargadas directamente por autoridades de mesas, siendo el 97\% del total. Del total de mesas:
15 mesas fueron cargadas entre las 20:00 hs y 21:20 hs, siendo el 47\%,
%20 mesas fueron cargadas entre las 20:00 hs y 21:40 hs, siendo el 63\%,
%22 mesas fueron cargadas entre las 20:00 hs y 21:50 hs, siendo el 69\% y 
%6 mesas fueron cargadas al día siguiente, siendo el 19\%.
1 mesa fue cargada 	por la Junta Electoral, siendo el 3\% del total.	
Ningún acta fue corregida por la Junta Electoral, lo que implica un 0\% de error en la carga realizada por las autoridades de mesa, o lo que es lo 	mismo un 100\% de datos correctamente cargados por las autoridades de mesas.

\subsection{Experiencia 2018}
\subsection{Experiencia 2019}

\subsection{Velocidad de carga}