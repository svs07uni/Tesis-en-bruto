\label{SistemaElectoral}
\chapter{Sistema Electoral}
En el sistema electoral existen partes del proceso vulnerables a la incorporación de recursos informáticos con el objetivo de agilizar su ejecución. Como concepto general, se considera voto electrónico a todo sistema informatizado para el acto de emitir y el recuento de votos en la mesa de votación. Pero además de este sistema existen variadas técnicas implementadas en distintos puntos del país y otros continentes que lograron ejecutarlo o que sólo lograron unos cuantos intentos para finalmente volver al sistema electoral tradicional.\newline
Se identifican 3 grandes mecanismos:
\begin{itemize}
    \item Los sistemas de recuento automático de votos mediante reconocimiento óptico de marcas aplicadas en la boleta por parte de los ciudadanos.	Los primeros sistemas datan del siglo XIX en Nueva York mediante tarjetas perforadas. En Venezuela (1994 – 2003) implementaron el sistema mediante boletas impresas en papel que luego el elector debía rellenar para luego ser contabilizado mediante reconocimiento óptico de caracteres \cite{eleccionesVenezuela}.\newline
    \underline{Ventajas:}Mantener el principio de que la voluntad del elector se mantiene en un trozo de papel anónimo fácilmente auditable independiente de los dispositivos y software usados.\newline
    \underline{Desventajas:}Suponen un doble trabajo para el votante (elegir y además controlar que su elección sea correctamente impresa). Pone en riesgo el anonimato del voto al poder agregar “suciedad” que en realidad codifique información que permita reconstruirse emisión de los votos y la relación de cada votante con su voto.\newline
    \underline{Conclusión:}Si bien cualquier sistema basado en papel podría ser adulterada, esto debe ser hecho individualmente con cada boleta, y el impacto de una persona corrupta se limita a las boletas bajo su custodia. En el sistema electrónico, en cambio, una única persona corrupta tiene el potencial de influenciar sobre un gran número de máquinas, comprometiendo la integridad de votos en masa, incluyendo los de las mesas cuyos fiscales actúen de buena fe.
    \item Los sistemas de registro electrónico directo (DRE) – urnas electrónicas. En Brasil, varios estados de EEUU y las últimas elecciones de Venezuela implementaron el sistema la cual realiza simultáneamente el registro y la tabulación del voto mediante un dispositivo informático, operado directamente por el votante (teclado, botonera o teclado táctil), registrándose el voto directamente en la memoria del dispositivo.\newline
    \underline{Ventajas:}Elimina por completo el uso de papel, no hay boletas que custodiar.\newline
    \underline{Desventajas:}No permite invalidar un voto o cometer errores clásicos que generan la anulación del voto.\newline
    \underline{Conclusión:}Genera un punto de tensión entre los ciudadanos que necesitan que el resultado refleje sus elecciones y los encargados de conducirlos que desean terminar la tarea con mayor rapidez y menor esfuerzo delegando la mayor responsabilidad que se pueda por posibles errores o actos de corrupción.
    \item Los sistemas de votación a distancia a través de Internet o Proyecto Debian es un proyecto comunitario con excelentes resultados de uso. El sistema es robusto, justo y difícil de engañar, pero solo funciona gracias al hecho de que el voto no es secreto.\newline
    \underline{Desventajas:}Identificar que un votante solo vote una única vez o impedir que vote a nombre de otra persona, o no esté habilitado para votar, a diferencia de verificar los documentos de identidad por parte de autoridades electorales. Por lo tanto, estos sistemas obligan a que la máquina que recibe el voto tenga conocimiento de quien lo está emitiendo, generando un punto de ataque para quien quiera violar el secreto del voto.
\end{itemize}

\section{Sistema electoral}

Un sistema electoral consiste de reglas y procedimientos técnicos que producen determinadas consecuencias, es decir, los sistemas electorales no son neutros. 
La complejidad es afectada por la cantidad de representantes que se eligen por distrito electoral y la forma de votación. El voto se realiza a través de las boletas electorales, las cuales son el instrumento por el cual el ciudadano expresa su voluntad, además de ser la prueba real del voto y el medio para realizar el recuento de votos o escrutinio.
Actualmente las condiciones jurídicas del sufragio están constituidas por su universalidad, igualdad, obligatoriedad y secreto. Sin embargo, en Argentina para llegar a ese punto, los condicionamientos y las formas de votar fueron modificándose a través de la historia \cite{historia}.\newline
Desde 1873 que el voto oral se convirtió en un voto escrito, fue modificandose hasta conseguir que en 1995, por la reforma de la Constitución Nacional queda hasta el día de la fecha el régimen electoral como el siguiente:
\begin{itemize}
    \item Sistema de Representación Proporcional
    \item Método de Conversión de votos en escaños: Sistema D'Hondt
    \item Umbral 3\% del padrón electoral de distrito
    \item La elección de todos los representantes argentinos se eligen en forma directa por el pueblo de la Nación Argentina.
\end{itemize}
\newline
El desarrollo de un sistema de votación electrónica es complejo, no sólo por sus desafíos técnicos, sino también por su importancia para el Estado y para la sociedad en general. Un sistema de voto electrónico involucra la consideración de aspectos de software, hardware, procesos operativos, y personas. En este sentido, cualquier desarrollo debe tener el aspecto de calidad del sistema como un objetivo primordial en el proceso.\newline
Según el Ministerio del Interior, Obras Públicas y Vivienda, se plantearon objetivos que un sistema de votación debe considerar:
\begin{itemize}
    \item Garantizar la completitud en la oferta electoral
    \item Simplificar el uso de boletas (por ej. con una boleta única)
    \item Brindar mayor accesibilidad a los ciudadanos a la hora de votar (por ej. para personas con alguna discapacidad)
    \item Lograr precisión y rapidez en el proceso de conteo de votos
\end{itemize}
Además un sistema de votación que incorpore algún grado de automatización debe construir la confianza de los ciudadanos, partidos políticos y gobierno, en el sistema y en el proceso de votación.\newline
A partir de las partes involucradas en el proceso surgen distintas propiedades a satisfacer:

\subsection{Secreto del voto}
Únicamente el votante debe poder tener conocimiento alguno del contenido del voto. La sola sospecha de que alguien pueda conocer el contenido de su voto impide la libre emisión del sufragio.
A pesar de eliminar el voto en cadena al utilizar urnas electrónicas, en el estado de Ohio se descubrió dos años después de haberlas usado, una grave falencia en estas urnas que permite reconstruir el vinculo entre el voto y el votante, a través de un reporte emitido por la urna al final del recuento.

\subsection{Integridad}
Requiere garantizar que la cadena de confianza no puede romperse. Esta propiedad también se encuentra ligada a la seguridad del sistema para proteger datos e información de accesos no autorizados, pero proveyendo al mismo tiempo acceso a personal autorizado para operar. En particular, se consideran importantes las propiedades de confidencialidad, integridad, disponibilidad y autenticidad. Los aspectos de Seguridad Informática cobran gran importancia en un sistema de misión crítica, un error en el sistema que pueda ser explotado por un atacante podría atentar contra alguno de los principios básicos del voto o el resultado de la votación en general. Frente a esto tenemos el ejemplo de las elecciones en Holanda que luego de 9 años un grupo de activistas demostró en un programa de televisión las vulnerabilidades del sistema. Esto generó que el gobierno decidiera volver al sistema tradicional en papel \cite{eleccionesHolanda}. Otro ejemplo serian las elecciones como las de EEUU en 2004, en las que las diferencias entre las encuestas en boca de urna y los resultados finales sugieren fuertemente que las urnas dieron resultados incorrectos.\newline
Esta propiedad incluye capturar la intención del voto de manera fehaciente (y sin introducir sesgos), registrar la intención de voto exactamente como fue capturada, garantizando que se respeta la voluntad de cada votante, es decir que el sistema no permita cambiar el voto una vez que el votante lo emitió, y contabilizar el voto exactamente como fue registrado. Si el sistema, por error o ataque, altera la suma de los votos individuales lo hará de una forma que será evidente para los ciudadanos.

\subsection{Capacidad de auditoría y control del proceso electoral (sin afectar los atributos de secreto e integridad anteriores)}
Capacidad de monitorear el sistema tanto en su diseño (estructura), como cuando se encuentra en funcionamiento (ejecución) y cuando ya dejó de utilizarse (análisis post-hoc). Un sistema de votación debe poderse auditar todos los niveles de hardware y software. El principio rector es que "las elecciones deben dar una evidencia consistente de un resultado preciso, aún cuando el rival sea quien escribe el software, administra la elección o gobierna el país".\newline
En el sistema de votación tradicional, la responsabilidad de auditoría está distribuida en todos porque todos pueden ver y entender el sistema. Lejos de aportar transparencia, la urna electrónica obstaculiza la capacidad de la mayoría de los ciudadanos de fiscalizar la elección, ya que queda necesariamente en manos de una élite tecnológica a la que el resto de la población no le queda otra que creerle. Las elecciones como las de EEUU en Georgia específicamente, en las que las elecciones fueron cien por ciento electrónicas y con el mismo tipo de máquinas. Esto quiere decir que gran parte de los votos no tiene recuento posible ya que no existe comprobante físico para contabilizar \cite{eleccionesGeorgia}.\newline
La incorporación de urnas electrónicas tiene efectos contrarios a este objetivo ya que las personas con poca afinidad con los sistemas computacionales (adultos mayores, personas de escasos recursos, con dificultad visual, entre otras) se verán enfrentados a un sistema mucho más complejo para votar. Por otra parte, las personas que auditan las elecciones (maestras de escuela, empleados públicos, fiscales de partidos políticos) se verán incapaces de auditar eficazmente este tipo de sistema. Sólo un grupo reducido de personas relacionadas al área de sistemas computacionales comprenderán el funcionamiento de estos sistemas, pero dificilmente se atreverán a firmar a conciencia una certificación de seguridad de las urnas pues no existe método formal de validación que los avale. Si bien no existen sistemas perfectos, la diferencia de impacto es sustancial. Un error mínimo en un sistema de votación electrónica puede alterar el resultado de una elección simultáneamente en un gran número de mesas.

\subsection{Igualdad de condiciones para todos los partidos políticos}
Todo sistema electrónico tiene un límite técnico donde manipular o visualizar la oferta electoral. Como ejemplo tenemos las elecciones del 2019 que se llevaron a cabo en Neuquén y Plottier para los comicios municipales. En estas elecciones se debieron distribuir 29 listas en una pantalla con capacidad de 18 casilleros. Lograr distribuir la totalidad de las listas implicó que no exista ventaja entre candidatos por las maniobras de mercantilizar la herramienta de las colectoras \cite{limiteColectoras}.

\subsection{Universalidad}
El sistema debe estar preparado para facilitar el sufragio de toda persona habilitada, esto incluye personas con requerimientos de accesibilidad, por ej. no videntes. Debemos tener en cuenta que toda persona posee los mismos derechos sobre el voto. Por lo tanto, si se requiere el acompañamiento debido a la complejidad o inaccesibilidad del sistema no se está alcanzando ninguna ventaja por sobre el sistema electoral tradicional.\newline
Como ejemplo, las máquinas usadas en elecciones electrónicas en general se basan en fotos y colores como medio de accesibilidad para las personas que no puedan leer o con baja visión. Por otro lado, en varios paises es común la instalación de un kit accesible como un teclado especial y auriculares para personas invidentes.

\subsection{Convalidación: análisis post-hoc del proceso electoral}
Una de las mayores ventajas es la rapidez en el conteo. De hecho, cuando todo sale bien, los resultados pueden ser inmediatos. El problema surge cuanto evaluamos el impacto potencial de las distintas cosas que pueden salir mal.\newline
La rapidez, sin confianza ni seguridad, no sirve para muchos en un proceso electoral. Esta es un área en la que la eficacia debe primar por sobre la eficiencia (rapidez).

\subsection{Usabilidad}
Debe ser fácil y adecuado para toda persona involucrada en el sistema, y cómo resulta el tipo de soporte brindado a sus usuarios. Un sistema electrónico debe ser fácil de aprender tanto para un votante como para las autoridades encargadas de dar soporte. Esta propiedad también ayuda al objetivo de rapidez ya que un sistema dificil de entender generará largas colas de espera para votar debido a que varias personas les lleva más tiempo entender y generar su sufragio. Al igual ocurriría con el tiempo de tardanza para el escrutinio.

\subsection{Cumplir con las normativas vigentes del proceso electoral}
Un sistema electoral regularmente no modifica sus políticas o reglas que lo conforman. De todos modos, un sistema de votación electrónica debe ser capaz de adaptarse a cualquier cambio adoptado por las autoridades de un pais, provincia o distrito.

\subsection{Integridad vs. Auditabilidad vs. Secreto}
Cabe destacar que para sistemas de votación (sea en formato de papel o electrónicos), se ha demostrado formalmente que existe una tensión o compromiso entre los atributos de integridad, auditabilidad y privacidad, y que existe una imposibilidad de satisfacer perfectamente los tres atributos en forma simultánea.\newline
Cualquier sistema de emisión electrónica de voto que busque solucionar los problemas inherentes a garantizar integridad y secreto, necesariamente será dificil de verificar formalmente y de auditar, incluso por expertos en la disciplina.

\section{Modelo de Referencia}
El problema de la votación se puede determinar en cinco fases secuenciales del proceso susceptible de ser automatizada. Las fases están derivadas del Código Electoral Nacional (Decreto nº2135, 1983)
\begin{enumerate}
    \item Emisión del voto
    \item Escrutinio de la mesa
    \item Generación de Documentos
    \item Comunicación de Resultados
    \item Procesamiento de Resultados y Publicación
\end{enumerate}

Como evaluación se analizan los riesgos y factibilidad técnica de las cuatro primeras fases, considerando distintas fuentes de información: atributos de calidad, los principios de construcción, los antecedentes de otros países, y la experiencia de los miembros de la Comisión.\newline
Se despliega considerando la fase menos riesgosa de implementar en el corto plazo, continuando progresivamente hasta llegar a la fase inicial considerada la más riesgosa y que por lo tanto requiere esfuerzos a largo plazo.
\subsection{Comunicación de Resultados} Asumiendo que se cuenta con un telegrama de escrutinio firmado (al menos) por la autoridad de mesa. A fin de evitar demoras en la transmisión de telegramas, es factible realizar su digitalización y transmisión desde el mismo local de votación (Sistema usado en las PASO). Desde el punto de vista de confiabilidad y verificabilidad del sistema, estos atributos aumentan si se transmite la imagen del documento, además de la información contenida en el soporte digital. De esta manera, si se publica tanto el resultado como la imagen del telegrama, cualquier ciudadano podrá contrastar la información y detectar posibles inconsistencias entre lo impreso y lo digital.\newline
    Disponer de un soporte digital en el documento evita la carga manual de los datos en el centro de procesamiento. Además, al recibirse la imagen del telegrama y los datos en formato digital, el resultado de la mesa puede ser publicado directamente sin intervención humana. Sin embargo, esta opción debe evaluarse con mucho cuidado, si la integridad de los datos fue afectada (es decir, el formato digital no coincide con lo impreso) o se dan otras situaciones anómalas (por ej. el telegrama no está firmado por la autoridad de mesa, o no se distinguen los números impresos) la confiabilidad del sistema puede verse afectada. Una solución de compromiso al problema descrito puede alcanzarse incorporando un proceso de validación de los datos.
\subsection{Generación de Documentos} Según el modelo de referencia, esta etapa comienza una vez que se dispone del resultado del escrutinio en la mesa. Al finalizar, se cuenta con actas de escrutinio, certificados de escrutinio y telegrama de escrutinio, todos estos firmados por autoridades de mesa. La incorporación de tecnología es factible, pudiendo realizar aportes en el desempeño de la generación de documentos y en una mayor oportunidad de verificación (o fiscalización) por parte de los partidos políticos. Aunque se observan riesgos asociados a la integridad de los documentos, ya que los formatos impresos pueden no coincidir con los formatos digitales, y la detección de estas diferencias puede darse luego de que se haya publicado el resultado provisorio de la mesa. Este tipo de riesgo no es fácil de mitigar, aunque no afectaría el resultado del escrutinio definitivo. También existen ciertos riesgos respecto a la disponibilidad del sistema.\newline
    Peligros identificados en relación al hardware (Máquina de generación de documentos):
    \begin{itemize}
        \item Alteración de documentos electrónicos en tránsito.
        \item La máquina no está operativa.
        \item La máquina no puede generar o imprimir los documentos.
        \item No es posible enviar los documentos electrónicos.
    \end{itemize}
    En esta etapa los riesgos inherentes están relacionados con el mecanismo de transmisión, por ello las máquinas que transfieren los documentos electrónicos no están sujetas a restriciones de seguridad tan estrictas como en las etapas anteriores.
    A partir de esto, se deben tener en cuenta aspectos técnicos mínimos que deberían ser considerados respecto del hardware para un dispositivo de conteo de votos.
    \begin{itemize}
        \item La máquina no debería poseer puertos de comunicación cableado o inalámbricos accesibles desde el exterior de la carcasa.
        \item La máquina debe contar con separación física entre memoria de datos y memoria de instrucciones de proceso.
     \end{itemize}  
        
\subsection{Conteo} Fase donde se totalizan los resultados por categoría y partido político, siendo el presidente de mesa el único encargado de realizar esta totalización. El circuito electoral requiere que cada mesa realice un recuento independiente y auditado que será plasmado en una o más actas con la misma información. En esta fase la computadora deberá ser un asistente en el conteo y que su resultado sea tomado como una primera aproximación. El resultado final plasmado en las actas será el obtenido por el conteo manual realizado por el presidente de mesa. De esta forma, se tiene un mecanismo robusto de conteo que es compatible con el principio de independencia del software. El conteo electrónico ayuda a las autoridades de mesa a generar confianza en el resultado del conteo, siempre y cuando se asegure el principio de independencia de software, esto es, la verificación efectiva de que la cuenta manual coincide con la cuenta electrónica, ya sea obligando a hacer el conteo manual o realizando risk limiting audits a posteriori.\newline
    Peligros identificados en relación al hardware (Máquina de Conteo de votos):
    \begin{itemize}
        \item La máquina no está operativa.
        \item La máquina es manipulada para cambiar su comportamiento o dañarla irreversiblemente.
    \end{itemize}
    A partir de esto, se deben tener en cuenta aspectos técnicos mínimos que deberían ser considerados respecto del hardware para un dispositivo de conteo de votos.
    \begin{itemize}
        \item La máquina no debería poseer puertos de comunicación cableado o inalámbricos accesibles desde el exterior de la carcasa.
        \item Filtrar toda conexión a la red eléctrica para evitar comunicaciones PLC. Se prefiere operación a baterías.
        \item La máquina debe contar con mecanismos de seguridad para evitar la descarga de datos que no contengan firma digital autorizada.
        \item La máquina debe contar con separación física entre memoria de datos y memoria de instrucciones de proceso.
        \item El fabricante de la MEB debe ser diferente al fabricante de la máquina de conteo de votos.
        \item Otro items más que se pueden encontrar en \cite{conicet}
    \end{itemize}
\subsection{Emisión del Voto} 
Fase en la que el ciudadano habilitado para votar expresa y registra su intención de voto. Cuando un dispositivo electrónico intermedia entre el votante y el registro de su intención de voto, el sistema electoral incluye un sistema electrónico para la emisión del voto, lo que popularmente se llama "voto electrónico". Los atributos de calidad referidos a un sistema de emisión de voto son: universalidad del voto, garantías de la oferta electoral, integridad (con sus 3 aspectos), confidencialidad, y usabilidad. La dificultad técnica, y que distingue al proceso de votación de otros sistemas informáticos, es que el requerimiento de mantener el secreto (un voto no puede ser asociado a su emisor) imposibilita luego explicar si un voto fue emitido válidamente por un votante o el mismo es consecuencia de un mal funcionamiento del software. Un sistema para la emisión del voto implica un alto riesgo, ya que está expuesto a todos los votantes habilitados, las autoridades de mesa y todo el sistema logístico del acto electoral. En una elección de escala nacional o incluso provincial, es un sistema distribuido de misión crítica en el que una falla (en todos o en un gran número de "nodos") puede ser catastrófica. Se debe considerar también que los problemas o errores que puedan introducirse en esta fase (detectables o no) se pueden propagar a las siguientes fases del modelo, y esto eleva el nivel de riesgo de todo el sistema.\newline

Peligros identificados en relación al hardware:
    \begin{itemize}
        \item La máquina emisora de boletas (MEB) se daña y no puede brindar servicio.
        \item La MEB es manipulada electrónicamente para modificar la representación del voto.
        \item La MEB es manipulada para almacenar y/o transmitir información adicional sobre la interacción del votante con la misma, que pueda ser utilizada posteriormente para asociar al elector con su voto.
        \item La MEB es manipulada para sesgar el modo de presentación de la oferta electoral.
        \item La MEB es reemplazada por hardware ilegítimo no distinguible por los usuarios.
    \end{itemize}
    Debido a esto, un hardware debe tener un diseño y construcción robusta, debiendo resistir manipulaciones maliciosas y vandalismo. \newline
    A partir de esto, se deben tener en cuenta aspectos técnicos mínimos que deberían ser considerados respecto del hardware para un dispositivo de emisión de votos.
    \begin{itemize}
        \item No debe tener capacidad de almacenamiento estático (disco rígido, SSD, etc.)
        \item El software debe ser de acción mínima y almacenable en Read Only Memory - One Time Programmable.
        \item Los componentes deben ser completamente inaccesiles, quedando solo disponible la interfaz del usuario.
        \item Filtrar toda conexión a la red eléctrica para evitar comunicaciones PLC. Se prefiere en cambio operación a baterías, duplicada en capacidad de energía como repaldo.
        \item No poseer memoria flash ni otro tipo de memoria no-volátil accesible en ejecución.
        \item La arquitectura debe contar con separación física (por hardware) entre memoria de datos y memoria de instrucciones de proceso (programa).
        \item Otro items más que se pueden encontrar en \cite{conicet}
    \end{itemize}
\end{enumerate}

\section{Costo en tecnología}
Uno de los principales factores para la selección de la tecnología a utilizar es la inversión necesaria. En el caso de hardware de uso específico para procesos de votación, su tasa de uso temporal es extremadamente baja pero se asume que se encuentra justificada en funcion de la importancia y seguridad del proceso. A esto debe sumarse el costo del depósito, traslado y custodia. \newline
El sistema usado para las elecciones provinciales en Neuquén para 2019 costó cerca de 100 millones de pesos, por la empresa MSA ganadora de la licitación. Este costo incluye capacitación del personal interviniente.\cite{eleccionesNeuquen}
